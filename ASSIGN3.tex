\documentclass[journal,12pt,twocolumn]{IEEEtran}

\usepackage{setspace}
\usepackage{gensymb}

\singlespacing


\usepackage[cmex10]{amsmath}

\usepackage{amsthm}
\usepackage{amsmath}
\usepackage{mathrsfs}
\usepackage{txfonts}
\usepackage{stfloats}
\usepackage{bm}
\usepackage{cite}
\usepackage{cases}
\usepackage{subfig}
\usepackage{amsmath,amssymb}

\usepackage{longtable}
\usepackage{multirow}

\usepackage{enumitem}
\usepackage{mathtools}
\usepackage{steinmetz}
\usepackage{tikz}
\usepackage{circuitikz}
\usepackage{verbatim}
\usepackage{tfrupee}
\usepackage[breaklinks=true]{hyperref}
\raggedbottom

\usepackage{tkz-euclide}
\usepackage{caption}
\usepackage{wrapfig}

\usetikzlibrary{calc,math}
\usepackage{listings}
    \usepackage{color}                                            %%
    \usepackage{array}                                            %%
    \usepackage{longtable}                                        %%
    \usepackage{calc}                                             %%
    \usepackage{multirow}                                         %%
    \usepackage{hhline}                                           %%
    \usepackage{ifthen}                                           %%
    \usepackage{lscape}     
\usepackage{multicol}
\usepackage{chngcntr}
\graphicspath{ {./images/} }

\DeclareMathOperator*{\Res}{Res}

\renewcommand\thesection{\arabic{section}}
\renewcommand\thesubsection{\thesection.\arabic{subsection}}
\renewcommand\thesubsubsection{\thesubsection.\arabic{subsubsection}}

\renewcommand\thesectiondis{\arabic{section}}
\renewcommand\thesubsectiondis{\thesectiondis.\arabic{subsection}}
\renewcommand\thesubsubsectiondis{\thesubsectiondis.\arabic{subsubsection}}


\hyphenation{op-tical net-works semi-conduc-tor}
\def\inputGnumericTable{}                                 %%

\lstset{
%language=C,
frame=single, 
breaklines=true,
columns=fullflexible
}
\begin{document}


\newtheorem{theorem}{Theorem}[section]
\newtheorem{problem}{Problem}
\newtheorem{proposition}{Proposition}[section]
\newtheorem{lemma}{Lemma}[section]
\newtheorem{corollary}[theorem]{Corollary}
\newtheorem{example}{Example}[section]
\newtheorem{definition}[problem]{Definition}

\newcommand{\BEQA}{\begin{eqnarray}}
\newcommand{\EEQA}{\end{eqnarray}}
\newcommand{\define}{\stackrel{\triangle}{=}}
\bibliographystyle{IEEEtran}
\providecommand{\mbf}{\mathbf}
\providecommand{\pr}[1]{\ensuremath{\Pr\left(#1\right)}}
\providecommand{\qfunc}[1]{\ensuremath{Q\left(#1\right)}}
\providecommand{\sbrak}[1]{\ensuremath{{}\left[#1\right]}}
\providecommand{\lsbrak}[1]{\ensuremath{{}\left[#1\right.}}
\providecommand{\rsbrak}[1]{\ensuremath{{}\left.#1\right]}}
\providecommand{\brak}[1]{\ensuremath{\left(#1\right)}}
\providecommand{\lbrak}[1]{\ensuremath{\left(#1\right.}}
\providecommand{\rbrak}[1]{\ensuremath{\left.#1\right)}}
\providecommand{\cbrak}[1]{\ensuremath{\left\{#1\right\}}}
\providecommand{\lcbrak}[1]{\ensuremath{\left\{#1\right.}}
\providecommand{\rcbrak}[1]{\ensuremath{\left.#1\right\}}}
\theoremstyle{remark}
\newtheorem{rem}{Remark}
\newcommand{\sgn}{\mathop{\mathrm{sgn}}}
\newcommand{\xrightarrowdbl}[2][]{%
    \leftarrow\mathrel{\mkern-14mu}\xrightarrow[#1]{#2}
}
% \providecommand{\abs}[1]{\left\vert#1\right\vert}
% \providecommand{\res}[1]{\Res\displaylimits_{#1}} 
% \providecommand{\norm}[1]{\left\lVert#1\right\rVert}
% %\providecommand{\norm}[1]{\lVert#1\rVert}
% \providecommand{\mtx}[1]{\mathbf{#1}}
% \providecommand{\mean}[1]{E\left[ #1 \right]}
\providecommand{\fourier}{\overset{\mathcal{F}}{ \rightleftharpoons}}
%\providecommand{\hilbert}{\overset{\mathcal{H}}{ \rightleftharpoons}}
\providecommand{\system}{\overset{\mathcal{H}}{ \longleftrightarrow}}
    %\newcommand{\solution}[2]{\textbf{Solution:}{#1}}
\newcommand{\solution}{\noindent \textbf{Solution: }}
\newcommand{\cosec}{\,\text{cosec}\,}
\providecommand{\dec}[2]{\ensuremath{\overset{#1}{\underset{#2}{\gtrless}}}}
\newcommand{\myvec}[1]{\ensuremath{\begin{pmatrix}#1\end{pmatrix}}}
\newcommand{\mydet}[1]{\ensuremath{\begin{vmatrix}#1\end{vmatrix}}}
\numberwithin{equation}{subsection}
\makeatletter
\@addtoreset{figure}{problem}
\makeatother
\let\StandardTheFigure\thefigure
\let\vec\mathbf
\renewcommand{\thefigure}{\theproblem}
\def\putbox#1#2#3{\makebox[0in][l]{\makebox[#1][l]{}\raisebox{\baselineskip}[0in][0in]{\raisebox{#2}[0in][0in]{#3}}}}
     \def\rightbox#1{\makebox[0in][r]{#1}}
     \def\centbox#1{\makebox[0in]{#1}}
     \def\topbox#1{\raisebox{-\baselineskip}[0in][0in]{#1}}
     \def\midbox#1{\raisebox{-0.5\baselineskip}[0in][0in]{#1}}
\vspace{3cm}
\title{ASSIGNMENT 3}
\author{NSV SARATH CHANDRA(CC20MTECH14001)}
\maketitle
\newpage
\bigskip
\renewcommand{\thefigure}{\theenumi}
\renewcommand{\thetable}{\theenumi}
    
\section{Problem}
Find the equation of the circle that passes through the points 
 \myvec{2\\3}, \myvec{3\\2}, \myvec{5\\1}.  

\section{Solution}

The general equation of circle is represented as 

\begin{align}
    \boldsymbol{{x^{T}}x}-2\boldsymbol{{c^{T}}x} + f = 0
\end{align}

where \textbf{c} is the center of the circle. Substituting the given points in the equation (2.0.1), we obtain
\begin{align}
2\myvec{2&3}\textbf{c} - f = 13
\end{align}
\begin{align}
2\myvec{3&2}\textbf{c} - f = 13
\end{align}
\begin{align}
2\myvec{5&1}\textbf{c} - f = 13
\end{align}

can be expressed in matrix form as 
\begin{align}
\myvec{4&6&-1\\6&4&-1\\10&2&-1}\myvec{\textbf{c}\\f} = \myvec{13\\13\\26}
\end{align}

The augmented matrix for (2.0.5) can be row reduced as follows

\begin{align}
\myvec{4&6&-1&13\\6&4&-1&13\\10&2&-1&26}
\end{align}

\begin{align}
\xleftrightarrow[R_2\leftarrow4R_3-10R_1]{R_3\leftarrow4R_3 - 10R_1}\myvec{4&6&-1&13\\0&-20&2&-26\\0&-52&6&-26}
\end{align}

\begin{align}
\xleftrightarrow[]{R_3\leftarrow5R_3-13R_2}\myvec{4&6&-1&13\\0&-20&2&-26\\0&0&4&208}
\end{align}

\begin{align}
\xleftrightarrow[R_1\leftarrow4R_1 + R_3]{R_2\leftarrow2R_2-R_3}\myvec{16&24&0&260\\0&-40&0&-260\\0&0&4&208}
\end{align}

\begin{align}
\xleftrightarrow[]{R_1\leftarrow5R_1+3R_2}\myvec{80&0&0&52\\0&-40&0&-260\\0&0&4&208}
\end{align}

\begin{align}
\xleftrightarrow[R_1\leftarrow\frac{R_1}{40}]{R_2\leftarrow\frac{R_2}{-20}, R_3\leftarrow\frac{R_1}{4}}\myvec{2&0&0&13\\0&2&0&13\\0&0&1&52}
\end{align}

From the matrix (2.0.11), 

\begin{align}
    \textbf{c} = \myvec{\frac{13}{2}\\\frac{13}{2}}
\end{align}

\begin{align}
    k = 52
\end{align}

\begin{align}
    r = \sqrt{{||\textbf{c}||}^2-f}=11
\end{align}

Hence the circle equation can be written as,
\begin{align}
    \boldsymbol{{x^{T}}x}-2\myvec{\frac{13}{2}&\frac{13}{2}}^T\boldsymbol{x} + 52 = 0
\end{align}
 
\begin{figure}[!]
\centering
\includegraphics[width=0.5\textwidth]{Figure_1}
\caption{Circle passing through the points A, B, C with center O}
\end{figure}
\end{document}